\documentclass[10pt,false,letter]{article}

\usepackage{fullpage}
\usepackage{setspace}
\usepackage{parskip}
\usepackage{titlesec}
\usepackage{xcolor}
\usepackage{lineno}





\PassOptionsToPackage{hyphens}{url}
\usepackage[colorlinks = true,
            linkcolor = blue,
            urlcolor  = blue,
            citecolor = blue,
            anchorcolor = blue]{hyperref}
\usepackage{etoolbox}
\makeatletter
\patchcmd\@combinedblfloats{\box\@outputbox}{\unvbox\@outputbox}{}{%
  \errmessage{\noexpand\@combinedblfloats could not be patched}%
}%
\makeatother


\usepackage[round]{natbib}
\let\cite\citep


\renewenvironment{abstract}
  {{\bfseries\noindent{\large\abstractname}\par\nobreak}}
  {}

\renewenvironment{quote}
  {\begin{tabular}{|p{13cm}}}
  {\end{tabular}}

\titlespacing{\section}{0pt}{*3}{*1}
\titlespacing{\subsection}{0pt}{*2}{*0.5}
\titlespacing{\subsubsection}{0pt}{*1.5}{0pt}


\usepackage{authblk}


\usepackage{graphicx}
\usepackage[space]{grffile}
\usepackage{latexsym}
\usepackage{textcomp}
\usepackage{longtable}
\usepackage{tabulary}
\usepackage{booktabs,array,multirow}
\usepackage{amsfonts,amsmath,amssymb}
\providecommand\citet{\cite}
\providecommand\citep{\cite}
\providecommand\citealt{\cite}
% You can conditionalize code for latexml or normal latex using this.
\newif\iflatexml\latexmlfalse
\providecommand{\tightlist}{\setlength{\itemsep}{0pt}\setlength{\parskip}{0pt}}%

\DeclareGraphicsExtensions{.pdf,.PDF,.png,.PNG,.jpg,.JPG,.jpeg,.JPEG}

\usepackage[utf8]{inputenc}
\usepackage[english]{babel}








\begin{document}

\title{FreeNAS - Development \& Community}



\author[1]{Devin M Whitney}%
\affil[1]{North Carolina State University}%


\vspace{-1em}



  \date{\today}


\begingroup
\let\center\flushleft
\let\endcenter\endflushleft
\maketitle
\endgroup










\section*{FreeNAS - Development \&
Community}\label{freenas---development-community}

FreeNAS is an open source software appliance that manages a network
attached storage device (NAS). It helps share files through a number of
protocols, including support for Windows, Mac OS X, and Unix of course.
it does not function as a normal operating system like Windows 10 or
Ubuntu, which helps it to be lightweight on the install drive. Plugins
are available that provide more features for the NAS, such as creating a
Minecraft server, Plex Media server, bit torrent downloader, and many
more. Different plugins can be made by the community, so the
possibilities are endless.

\begin{itemize}
\tightlist
\item
  FreeNAS is an active project licensed under the BSD License which is
  approved by the OSI and FSF
\item
  There is a website that allows you to submit a bug and view changes
\item
  Feature requests are made in the FreeNAS community forums
\item
  The latest source code is available on the FreeNAS GitHub
\item
  You can learn about contributing to the project, and contact the
  developers through the community forum
\item
  iX Systems provides FreeNAS training through ``iX University'' online
\item
  iX Systems has commercial support for the storage devices that they
  create
\end{itemize}

www.freenas.org

\selectlanguage{english}
\clearpage
\end{document}

